%%%%%%%%%%%%%%%%%%%%%%%%%%%%%%%%%%%%%%%%%
% Friggeri Resume/CV
% XeLaTeX Template
% Version 1.2 (3/5/15)
%
% This template has been downloaded from:
% http://www.LaTeXTemplates.com
%
% Original author:
% Adrien Friggeri (adrien@friggeri.net)
% https://github.com/afriggeri/CV
%
% This version has been modified by:
% Sebastian Dziadzio (sebastian.dziadzio@gmail.com) 
%
% License:
% CC BY-NC-SA 3.0 (http://creativecommons.org/licenses/by-nc-sa/3.0/)
%%%%%%%%%%%%%%%%%%%%%%%%%%%%%%%%%%%%%%%%%

\documentclass[]{friggeri-cv_osx}

\usepackage{fontspec}
\usepackage{fontawesome}

\begin{document}
\header{sebastian}{dziadzio}{machine learning engineer}


% sidebar
\begin{aside}
\section{contact}
\href{http:sebastiandziadzio.com}{\color{gray} \faHome} \href{https://github.com/sebastiandziadzio}{\color{gray} \faGithub} \href{https://twitter.com/sebadzia}{\color{gray} \faTwitter} \href{http://pl.linkedin.com/in/sebastiandziadzio}{\color{gray} \faLinkedin} 
+49 17627954502
\href{mailto:sebastian.dziadzio@gmail.com}{sebastian.dziadzio@gmail.com}
\section{software}
Python, C++, Keras
TensorFlow, Spark, R
\section{knowledge}
scientific research
machine learning
signal processing
software development
\section{languages}
Polish - native
English - fluent
Spanish - good
German - good
\end{aside}


\section{education}
\begin{entrylist}
\entry
{2013-2015}
{Master of Computer Science, Intelligent Systems}
{AGH University}
{Thesis: Application of Morphosyntactic and Semantic Language Models in Automatic Speech Recognition.\\
Key courses: advanced algorithms, robotics and mechatronics, machine learning, natural language processing, computational intelligence.\\}

\entry
{2010-2013}
{Bachelor of Acoustical Engineering, Vibroacoustics}
{AGH University}
{Thesis: Unit Selection Text to Speech System for Polish.\\
Key courses: algebra, calculus, physics, mechanics, object-oriented programming, signal processing, robotics, image processing.\\}
\end{entrylist}


\section{experience}
\begin{entrylist}
\entry
{2016--Now}
{Software Engineer, Cliqz}
{Munich, Germany}
{Tech: Python, Keras, TensorFlow, Spark, R.\\I currently lead a project investigating the use of deep recurrent neural networks in search ranking. I built a data pipeline for collecting, processing, analysing, and visualising telemetry measurements.\\}

\entry
{2015--2016}
{Software Engineer, Nokia Networks}
{Cracow, Poland}
{Tech: C++, Python.\\I implemented new features for the LTE base transceiver stations. I designed unit tests, system component tests, and integration tests.\\}

\entry
{2014--2015}
{Junior Researcher, AGH University}
{Cracow, Poland}
{Tech: Python, C++, Ruby.\\I conducted a comparative analysis of morphosyntactic language models in automatic speech recognition and published the results in a conference paper. I also investigated the use of semantic models in interactive voice response systems and co-authored a paper on context modeling.\\}
\end{entrylist}

\section{publications}
\begin{entrylist}
\entry
{2015}
{Comparing Language Models Trained on Written Texts and Speech Transcripts}
{B. Ziółko, S. Dziadzio, A. Nabożny, A. Pohl}
{Proceedings of the 10th International Symposium Advances in Artificial Intelligence and Applications\\}

\entry
{2015}
{ContextViewer – Tool for Visualisation and Processing of Mobile Sensors Data}
{S. Bobek, S. Dziadzio, P. Jaciów, M. Ślażyński, G. Nalepa}
{Proceedings of the 9th International and Interdisciplinary Conference on Modeling and Using Context\\\\}
\end{entrylist}


\section{awards}
\begin{entrylist}
\entry
{2010--2012}
{Chancellor's Scholarship}
{AGH University}
{Awarded annually based on academic record.}
\end{entrylist}


\section{extracurricular courses}
\begin{entrylist}
\entry
{2016}
{Cliqz University}
{Cliqz}
{2-week workshop on programming and machine learning.}
\entry
{2016}
{Advanced Python}
{Nokia}
{3-day intensive workshop.}
\entry
{2016}
{C++ Standard Template Library}
{Nokia}
{3-day intensive workshop.}
\entry
{2015}
{Clean Code}
{Nokia}
{8-week online course on software development principles.}
\entry
{2014}
{Discrete Time Systems and Signals}
{RiceX}
{5-week online course, 8 hours per week.}
\entry
{2014}
{Machine Learning}
{Coursera}
{11-week online course, 8 hours per week.}
\entry
{2013}
{Learning from Data}
{CaltechX}
{10-week online course, 20 hours per week.}
\entry
{2012}
{Artificial Intelligence}
{BerkeleyX}
{12-week online course, 10 hours per week.}
\entry
{2012}
{Circuits and Electronics}
{MITx}
{5-week online course, 6 hours per week.}
\end{entrylist}


\section{presentations}
\begin{entrylist}
\entry
{2017}
{Introduction to Unix and Python}
{Cliqz}
{A monthly workshop for new hires.}
\entry
{2016}
{Language Models for Automatic Speech Recognition}
{Cliqz}
{A one-time lecture on language modelling.}
\entry
{2014}
{Unit Selection Speech Synthesizer for Polish}
{XXII Pacific Voice Conference}
{Student presentation.}
\end{entrylist}


\end{document}
