%%%%%%%%%%%%%%%%%%%%%%%%%%%%%%%%%%%%%%%%%
% Friggeri Resume/CV
% XeLaTeX Template
% Version 1.2 (3/5/15)
%
% This template has been downloaded from:
% http://www.LaTeXTemplates.com
%
% Original author:
% Adrien Friggeri (adrien@friggeri.net)
% https://github.com/afriggeri/CV
%
% This version has been modified by:
% Sebastian Dziadzio (sebastian.dziadzio@gmail.com) 
%
% License:
% CC BY-NC-SA 3.0 (http://creativecommons.org/licenses/by-nc-sa/3.0/)
%%%%%%%%%%%%%%%%%%%%%%%%%%%%%%%%%%%%%%%%%

\documentclass[]{friggeri-cv}

\usepackage{fontspec}
\usepackage{fontawesome}

\begin{document}
\header{sebastian}{dziadzio}{software engineer}


% sidebar
\begin{aside}
\section{contact}
+49 17627954502
\href{mailto:sebastian.dziadzio@gmail.com}{sebastian.dziadzio@gmail.com}
\href{http:sebastiandziadzio.com}{\color{gray} \faHome} \href{https://github.com/sebastiandziadzio}{\color{gray} \faGithub} \href{https://twitter.com/sebadzia}{\color{gray} \faTwitter} \href{http://pl.linkedin.com/in/sebastiandziadzio}{\color{gray} \faLinkedin} 
\section{software}
Python, R, Keras
TensorFlow, Spark
\section{knowledge}
machine learning
speech recognition
software development
\section{languages}
Polish - native
English - fluent
Spanish - intermediate
German - intermediate
\end{aside}


\section{education}
\begin{entrylist}
\entry
{2013-2015}
{Master of Computer Science}
{AGH University}
{Intelligent Systems \\ Key courses: advanced algorithms, computational intelligence, machine learning, natural language processing.\\}

\entry
{2010-2013}
{Bachelor of Acoustical Engineering}
{AGH University}
{Vibroacoustics \\ Key courses: algebra, calculus, digital signal processing, object-oriented programming, speech technology.}
\end{entrylist}


\section{experience}
\begin{entrylist}
\entry
{2016--Now}
{Cliqz}
{Munich, Germany}
{Software Engineer \\
  I am investigating the use of deep recurrent neural networks in search ranking. Previously I was responsible for building a data pipeline for collecting, processing, analysing, and visualising telemetry measurements. In my daily work I use Python, Keras, TensorFlow, Spark, and Luigi. \\}  

\entry
{2015--2016}
{Nokia Networks}
{Cracow, Poland}
{Software Engineer \\
  I was responsible for implementing new features for the LTE control plane~(C++), unit testing~(Google Test), and~integration testing (Python). I~worked in an agile environment (Scrum).\\}

\entry
{2014--2015}
{AGH University}
{Cracow, Poland}
{Junior Researcher \\
  I conducted a comparative analysis of morphosyntactic language models in automatic speech recognition and published the results in a conference paper. I also investigated the use of semantic models in interactive voice response systems and co-authored a paper on context modeling.\\}
\end{entrylist}

%\section{awards}
%\begin{entrylist}
%\entry
%{2010--2012}
%{Chancellor's Scholarship}
%{AGH University}
%{Awarded annually based on academic record.}
%\end{entrylist}


\section{publications}
\begin{entrylist}
\entry
{2015}
{Comparing Language Models Trained on Written Texts and Speech Transcripts}
{B. Ziółko, S. Dziadzio, A. Nabożny, A. Pohl}
{Proceedings of 10th International Symposium Advances in Artificial Intelligence and Applications\\}

\entry
{2015}
{ContextViewer – Tool for Visualisation and Processing of Mobile Sensors Data}
{S. Bobek, S. Dziadzio, P. Jaciów, M. Ślażyński, G. Nalepa}
{Proceedings of 9th International and Interdisciplinary Conference on Modeling and Using Context}
\end{entrylist}

\end{document}
