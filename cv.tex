%%%%%%%%%%%%%%%%%%%%%%%%%%%%%%%%%%%%%%%%%
% Friggeri Resume/CV
% XeLaTeX Template
% Version 1.2 (3/5/15)
%
% This template has been downloaded from:
% http://www.LaTeXTemplates.com
%
% Original author:
% Adrien Friggeri (adrien@friggeri.net)
% https://github.com/afriggeri/CV
%
% This version has been modified by:
% Sebastian Dziadzio (sebastian.dziadzio@gmail.com) 
%
% License:
% CC BY-NC-SA 3.0 (http://creativecommons.org/licenses/by-nc-sa/3.0/)
%%%%%%%%%%%%%%%%%%%%%%%%%%%%%%%%%%%%%%%%%

\documentclass[]{friggeri-cv_osx}

\usepackage{fontspec}
\usepackage{fontawesome}

\begin{document}
\header{sebastian}{dziadzio}{machine learning engineer}

\begin{center}
\href{http:sebastiandziadzio.com}{\color{gray} \Large \faHome} \hspace{0.05cm}
\href{https://github.com/sebastiandziadzio}{\color{gray} \Large \faGithub} \hspace{0.05cm}
\href{mailto:sebastian.dziadzio@gmail.com}{\color{gray} \Large\faEnvelope} \hspace{0.05cm}
\href{https://twitter.com/sebadzia}{\color{gray} \Large\faTwitter} \hspace{0.05cm}
\href{http://pl.linkedin.com/in/sebastiandziadzio}{\color{gray} \Large\faLinkedin} \hspace{0.05cm}
\end{center}
%\vspace{0.5cm}

\section{education}
\begin{entrylist}
\entry
{2014-2016}
{Master of Computer Science, Intelligent Systems}
{AGH University}
{Key courses: natural language processing, advanced algorithms, semantic web, R\&D seminar, machine learning, multimodal interfaces, computational intelligence.\\
Thesis: Application of Morphosyntactic and Semantic Language Models in Automatic Speech Recognition.\\
Final grade: 5/5 (very good).\\}

\entry
{2010-2014}
{Bachelor of Acoustical Engineering, Vibroacoustics}
{AGH University}
{Key courses: algebra, calculus, speech technology, physics, object-oriented system design, signal processing, cognitive robotics, numerical computation and simulation.\\
Thesis: Unit Selection Text to Speech System for Polish.\\
Final grade: 4.5/5 (good plus).\\}

\end{entrylist}


\section{experience}
\begin{entrylist}
\entry
{2016-Now}
{Software Engineer, Cliqz}
{Munich, Germany}
{I lead a project investigating the use of word embeddings and LSTM networks in information retrieval. Previously I built a data pipeline for collecting, processing, analysing, and visualising telemetry data. I use Python, Keras, TensorFlow, Spark, and R.\\}

\entry
{2015-2016}
{Software Engineer, Nokia Networks}
{Cracow, Poland}
{I implemented new features for the LTE base transceiver stations. I designed unit tests, system component tests, and integration tests. I used C++ and Python.\\}

\entry
{2014-2015}
{Junior Researcher, AGH Signal Processing Group}
{Cracow, Poland}
{I investigated the use of semantic models in interactive voice response systems as a part of a research grant on human-computer interaction. I used C++ and Python.\\}

\entry
{2013}
{Intern, AGH Signal Processing Group}
{Cracow, Poland}
{I conducted a comparative analysis of language models in automatic speech recognition and published the results in a conference paper. I used Python and Ruby.\\}
\end{entrylist}


\section{knowledge}
\begin{entrylist}
\entry
{}
{Programming}
{}
{Python, C++, Keras, TensorFlow, Spark, R}
\entry
{}
{Expertise}
{}
{machine learning, natural language processing, software development}
\entry
{}
{Skills}
{}
{presentation, scientific writing, implementing ideas from research papers}
\entry
{}
{Languages}
{}
{Polish (native), English (fluent), Spanish (good), German (conversational)}
\end{entrylist}


\section{publications}
\begin{entrylist}
\entry
{2015}
{Comparing Language Models Trained on Written Texts and Speech Transcripts\\}
{B. Ziółko, S. Dziadzio, A. Nabożny, A. Pohl}
{Proceedings of the 10\textsuperscript{th} International Symposium on Advances in Artificial Intelligence and Applications\\}

\entry
{}
{ContextViewer – Tool for Visualisation and Processing of Mobile Sensors Data\\}
{S. Bobek, S. Dziadzio, P. Jaciów, M. Ślażyński, G. Nalepa}
{Proceedings of the 9\textsuperscript{th} International and Interdisciplinary Conference on Modeling and Using Context\\}
\end{entrylist}


\section{extracurricular courses}
\begin{entrylist}
\entry
{2016}
{Cliqz University}
{Cliqz}
{2-week workshop on programming and machine learning}
\entry
{}
{Advanced Python}
{Nokia}
{3-day intensive workshop}
\entry
{}
{C++ Standard Template Library}
{Nokia}
{3-day intensive workshop}
\entry
{2015}
{C++ Programming}
{Nokia}
{4-week course, 16 hours per week}
\entry
{2014}
{Discrete Time Systems and Signals}
{RiceX}
{5-week online course, 8 hours per week}
\entry
{}
{Machine Learning}
{Coursera}
{11-week online course, 8 hours per week}
\entry
{2013}
{Learning from Data}
{CaltechX}
{10-week online course, 20 hours per week}
\entry
{2012}
{Artificial Intelligence}
{BerkeleyX}
{12-week online course, 10 hours per week}
\entry
{}
{Circuits and Electronics}
{MITx}
{5-week online course, 6 hours per week\\}
\end{entrylist}


\section{presentations}
\begin{entrylist}
\entry
{2017}
{Introduction to Unix}
{Cliqz}
{A monthly workshop for new hires}
\entry
{2017}
{Introduction to Python}
{Cliqz}
{A one-time workshop for new hires}
\entry
{2016}
{Language Models for Automatic Speech Recognition}
{Cliqz}
{A one-time lecture on language modeling}
\entry
{2014}
{Unit Selection Speech Synthesizer for Polish}
{22\textsuperscript{nd} Annual Pacific Voice Conference}
{Student paper}
\entry
{2013}
{Building a Corpus for a TTS System}
{OSKA Student Conference}
{Student presentation}
\entry
{}
{Neologisms, loanwords, and pidgins}
{50\textsuperscript{th} AGH SKN Conference}
{1\textsuperscript{st} prize in humanities}
\entry
{2011}
{A Distributed Platform for Radiation Measurements}
{52\textsuperscript{nd} AGH Scientific Session}
{Student presentation}
\end{entrylist}


\section{awards}
\begin{entrylist}
\entry
{2011--2012}
{Chancellor's Scholarship}
{AGH University}
{Awarded annually based on academic record}
\end{entrylist}


\section{projects}
\begin{entrylist}
\entry
{2016}
{Alpy}
{Group for Machine Learning Research at Jagiellonian University}
{Active learning module for Scikit-Learn}

\entry
{2015}
{ContextViewer}
{Group for Engineering Intelligent Systems and Technologies at AGH}
{A web-based tool for preprocessing, visualisation, and browsing of contextual data}

\entry
{2014}
{Polish synsets}
{Signal Processing Group at AGH}
{C++ API for the Polish Wordnet}

\entry
{2014}
{Unit selection speech synthesis}
{Digital Signal Processing Student Group}
{Recording and labelling data for a text-to-speech system}
\end{entrylist}

\end{document}
